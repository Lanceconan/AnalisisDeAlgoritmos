
\section{Aplicaciones y Ventajas/Desventajas del uso de Algoritmo de Búsqueda Binaria Recursiva}

\subsection{Ventajas}

\begin{itemize}
\item Es un método eficiente siempre que el vector se encuentre ordenado.
\item Proporciona un medio para reducir el tiempo requerido para buscar en una lista.
\item Es más rápido debido a su recursividad, su mayor ventaja es con los archivos extensos.
\item El código del procedimiento de esta búsqueda es corto en comparación con las demás técnicas de búsquedas.
\end{itemize}

\subsection{Desventajas}

\begin{itemize}
\item El archivo debe estar ordenado y el almacenamiento de un archivo suele plantear problemas en la inserción y eliminación de elementos.
\item No revisa todos los elementos del archivo.
\item Debe conocerse el número de elementos.
\item Uso de memoria por recursividad.
\end{itemize}

\subsection{Aplicaciones}

\begin{itemize}
\item Partición Binaria del Espacio, usada en muchos videojuegos 3D para determinar que objeto necesita ser renderizado.
\item Binary Tries, usada en casi todos los router de alta banda ancha para guardar las tablas de enrutamiento.
\item Codificación Huffman, algoritmo usado para la compresión de datos, se utiliza en los formatos \texttt{.jpeg} y \texttt{.mp3.}
\end{itemize}

