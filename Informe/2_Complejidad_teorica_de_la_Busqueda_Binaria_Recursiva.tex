
\section{Complejidad teórica de la Búsqueda Binaria Recursiva}

A diferencia de la búsqueda binaria iterativa, su versión recursiva es más lenta con cada incremento de número de elementos, ya que existirán más llamadas a la función por resolver, con el consiguiente gasto de tiempo de guardar y restaurar parámetros.

\textbf{Mejor caso}: la búsqueda binaria coincide con el elemento buscado en el primer punto medio: sólo se necesitaría una comparación de elementos. Esto significa que sus tiempos de ejecución óptimos no dependen de la cantidad de datos: son constantes y por tanto proporcionales a 1, es decir, son de O(1).

\textbf{Peor caso}: En el peor caso la búsqueda binaria recursiva el valor a encontrar no se encuentra en el arreglo tras haber realizado las llamadas correspondientes del algoritmo (dividir y comparar), requiriendo un nivel de orden O(log $n$).

\textbf{Caso promedio}: Se debe tomar en cuenta el comportamiento del algoritmo y su condición de término, obteniéndose la siguiente ecuación de recurrencia:

\begin{center}
$T(n) = T(n/2) + C$\\
$T(1) = 1$
\end{center}

Donde ''$n/2$'' responde al proceso de comparar el elemento medio del arreglo con el valor a buscar y el T(1) = 1 resume el mejor caso: el dato es encontrado en centro de la estructura mediante una llamada.

\textbf{Para hallar el nivel de complejidad promedio se hacen los siguientes pasos:}

\begin{enumerate}[I. ]

\item Reemplazar las $n$ en la ecuación original con valores sucesivos:

\begin{center}
$T(n/2) = T(n/4) + C$\\
$T(n/4) = T(n/8) + C$\\
$T(n/8) = T(n/16) + C$\\
$T(n/16) = T(n/32) + C$
\end{center}

\item Generalizar en términos de $n$ y $k$ para eliminar la recurrencia:

\begin{center}
$T(n) = T(n/2) + C$\\
$T(n) = T(n/4) + C + C$\\
$T(n) = T(n/8) + C + C + C$\\
$T(n) = T(n/16) + C + C + C + C$\\
$T(n) = T(n/32) + C + C + C + C + C$
\end{center}

Nótese que al realizar las sucesiones se llega dar con el siguiente patrón: que los denominadores de las fracciones obedecen a potencias de 2 ($2^1, 2^2, 2^3, 2^4, $etc...) y que las constantes ''$C$'' son directamente proporcionales a los valores arrojados en los exponentes. Por lo que al condensar todo en una fórmula se obtiene que:

{\centerline{ \bfseries $T(n) = T(n/2^k) + kC$} }

\item Igualar con la condición inicial los términos encerrados en T($n$)

PREVIO: se debe eliminar la variable ''k'' con la condición de tope:

{\centerline{ \textbf{$n/2^k = 1 \rightarrow n = 2^k$}} }

Se aplica propiedades de logaritmos para dejar la ecuación en términos de ''n'' :

{\centerline{ \textbf{log$_2 n = k$}} }

{\centerline{ \textbf{Resultado: $T(n) = T( (n/2)^{log_2 n}) + log2n * C$}} }

Usando la condición de tope T(1) = 1

{\centerline{ \textbf{$T(n) = 1 + log_2 n * C$}} }

Con lo obtenido recientemente se llega a la conclusión de que para el caso promedio la Búsqueda Binaria Recursiva tiene complejidad de orden \textbf{O(log$_2 n$)}.




\end{enumerate}